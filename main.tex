\documentclass[]{fim-uhk-thesis}
%\documentclass[english]{fim-uhk-thesis} % without assignment - for the work start to avoid compilation problem
% * Je-li práce psaná v anglickém jazyce, je zapotřebí u třídy použít 
%   parametr english následovně:
%   If thesis is written in English, it is necessary to use 
%   parameter english as follows:
%      \documentclass[english]{fim-uhk-thesis}
% * Je-li práce psaná ve slovenském jazyce, je zapotřebí u třídy použít 
%   parametr slovak následovně:
%   If the work is written in the Slovak language, it is necessary 
%   to use parameter slovak as follows:
%      \documentclass[slovak]{fim-uhk-thesis}
% * Je-li práce psaná v anglickém jazyce se slovenským abstraktem apod., 
%   je zapotřebí u třídy použít parametry english a enslovak následovně:
%   If the work is written in English with the Slovak abstract, etc., 
%   it is necessary to use parameters english and enslovak as follows:
%      \documentclass[english,enslovak]{fim-uhk-thesis}

% Základní balíčky jsou v souboru šablony fim-uhk-thesis.cls
% Basic packages are at template file fim-uhk-thesis.cls
% zde můžeme vložit vlastní balíčky / you can place own packages here



%---rm---------------
\renewcommand{\rmdefault}{lmr}%zavede Latin Modern Roman jako rm / set Latin Modern Roman as rm
%---sf---------------
\renewcommand{\sfdefault}{qhv}%zavede TeX Gyre Heros jako sf
%---tt------------
\renewcommand{\ttdefault}{lmtt}% zavede Latin Modern tt jako tt

% vypne funkci šablony, která automaticky nahrazuje uvozovky,
% aby nebyly prováděny nevhodné náhrady v popisech API apod.
% disables function of the template which replaces quotation marks
% to avoid unnecessary replacements in the API descriptions etc.
\csdoublequotesoff

\usepackage{url}

% =======================================================================
% Balíček "hyperref" vytváří klikací odkazy v pdf, pokud použijeme pdflatex.
% Problém je, že balíček hyperref musí být uveden jako poslední, takže nemůže
% být v šabloně.
% "hyperref" package create clickable links in pdf if you are using pdflatex.
% Problem is that this package have to be introduced as the last one so it 
% can not be placed in the template file.

  \usepackage{color}
  \usepackage[unicode,colorlinks,hyperindex,plainpages=false,urlcolor=black,linkcolor=black,citecolor=black]{hyperref}
  \definecolor{links}{rgb}{0,0,0}
  \definecolor{anchors}{rgb}{0,0,0}
  \def\AnchorColor{anchors}
  \def\LinkColor{links}
  \def\pdfBorderAttrs{/Border [0 0 0] } % bez okrajů kolem odkazů / without margins around links
  \pdfcompresslevel=9

% Řešení problému, kdy odkazy na obrázky vedou za obrázek
% This solves the problems with links which leads after the picture
\usepackage[all]{hypcap}

\projectinfo{
  % Práce / Thesis
  project={DEF},            % typ práce BP/SP/DP/DR / thesis type (SP = term project)
  year={měsíc rok},             % rok odevzdání / year of submission
  date=\today,             % datum odevzdání / submission date
  % Název práce / Thesis title
  title.cs={Název bakalářské/diplomové práce},  % název práce v češtině či slovenštině (dle zadání) / thesis title in czech language (according to assignment)
  title.en={Název práce v anglickém jazyce}, % název práce v angličtině / thesis title in english
  subtitle={(podtitul práce)},   % podtitul práce (nepovinné) / subtitle (optional)
  % Autor / Author
  author.name={Jméno},   % jméno autora / author name
  author.surname={Příjmení},   % příjmení autora / author surname 
  author.field={Studijní obor},   % obor
  %author.title.p={}, % titul před jménem (nepovinné) / title before the name (optional)
  %author.title.a={}, % titul za jménem (nepovinné) / title after the name (optional)
  % Katedra / Department
  department={DEF}, % zkratka katedry / abbreviation of the department
  % Školitel / Supervisor
  supervisor.name={Jméno},   % jméno školitele / supervisor name 
  supervisor.surname={Příjmení},   % příjmení školitele / supervisor surname
  supervisor.title.p={Titul},   % titul před jménem (nepovinné) / title before the name (optional)
  supervisor.title.a={},    % titul za jménem (nepovinné) / title after the name (optional)
  % Klíčová slova / Keywords
  keywords.cs={(typicky 5 – 6 slovních spojení)}, % klíčová slova v českém či slovenském jazyce / keywords in czech or slovak language
  keywords.en={}, % klíčová slova v anglickém jazyce / keywords in english
  % Abstrakt / Abstract
  annotation.cs={Text abstraktu – shrnutí cíle, významu práce a výsledky v ní dosažené. Délka minimálně 100 a maximálně 200 slov.},
  annotation.en={Abstrakt v anglickém jazyce. Délka minimálně 100 a maximálně 200 slov.},
  % Prohlášení (u anglicky psané práce anglicky, u slovensky psané práce slovensky) / Declaration (for thesis in english should be in english)
  declaration={Prohlašuji, že jsem bakalářskou/diplomovou práci zpracoval/zpracovala samostatně a s použitím uvedené literatury.},
  %declaration={I hereby declare that this Bachelor's thesis was prepared as an original work by the author under the supervision of Mr. X
% The supplementary information was provided by Mr. Y
% I have listed all the literary sources, publications and other sources, which were used during the preparation of this thesis.},
  % Poděkování (nepovinné, nejlépe v jazyce práce) / Acknowledgement (optional, ideally in the language of the thesis)
  acknowledgment={Děkuji vedoucímu bakalářské/diplomové práce titul, jméno, příjmení za metodické vedení práce a... .}
  %acknowledgment={Here it is possible to express thanks to the supervisor and to the people which provided professional help
%(external submitter, consultant, etc.).}
}

% řeší osamocený první/poslední řádek odstavce na předchozí/následující stránce
% solves first/last row of the paragraph on the previous/next page
\clubpenalty=10000
\widowpenalty=10000


% začátek dokumentu
\begin{document}
  % Vysázení titulních stran / Typesetting of the title pages
  % ----------------------------------------------
  \maketitle
  
  % Obsah / Content
  % ----------------------------------------------
  \thispagestyle{empty}
  \setcounter{tocdepth}{3}
  {\hypersetup{hidelinks}\tableofcontents}
  
  % Seznam obrázků a tabulek (pokud práce obsahuje velké množství obrázků/tabulek)
  % List of figures and list of tables (if the thesis contains a lot of pictures/tables)
  
  \ifczech
    \renewcommand\listfigurename{Seznam obrázků}
  \fi
  \ifslovak
    \renewcommand\listfigurename{Zoznam obrázkov}
  \fi
   {\hypersetup{hidelinks}\listoffigures}
   \thispagestyle{empty}
  
  \ifczech
    \renewcommand\listtablename{Seznam tabulek}
  \fi
  \ifslovak
    \renewcommand\listtablename{Zoznam tabuliek}
  \fi
  
  {\hypersetup{hidelinks}\listoftables}
  \thispagestyle{empty}

  % vynechání stránky v oboustranném režimu
  % Skip the page in the two-sided mode
  \iftwoside
    \cleardoublepage
  \fi

  % Text práce / Thesis text
  % ----------------------------------------------
  \setcounter{page}{0}
  % ################################
\section{Úvod}

Zde vysvětlit problémovou situaci a otázky, které se budou v bakalářské/diplomové práci řešit.


% ################################
\section{Cíl a metodika práce}

Smysl a účel, výzkumné otázky.  

Cíle, hypotézy/výzkumné otázky, způsob hledání odpovědí na výzkumné otázky včetně metodiky vlastního výzkumu/šetření, literární rešerše.


% ################################
\section{Kapitola -- Vlastní text práce}

Vlastní řešení dokládá student zpravidla v několika kapitolách. Podle charakteru práce musí student uvážit, zda informace netextové povahy (data, tabulky, obrázky atd.) bude uvádět přímo v textu, nebo je zařadí až za celou práci ve formě příloh, či bude kombinovat oba způsoby. 

Více podrobností viz Metodické pokyny pro vypracování bakalářských a diplomových prací (zveřejňované formou výnosů děkana) a v kurzu MES -- Metodologický seminář. \cite{autor00}

\subsection{Podkapitola}

Vlastní text práce.

\begin{table}[htb!]  
\caption{Název tabulky.}

\begin{tabular}{| l | l | l | l |}
\hline
 \hspace{0.22\textwidth} & \hspace{0.22\textwidth} & \hspace{0.22\textwidth} & \hspace{0.22\textwidth} \\
\hline
 & & & \\
\hline
 & & & \\
\hline
 & & & \\
\hline
\end{tabular} \\

Zdroj: citace zdroje, nebo autor, vlastní zpracování
%\label{tab:empty}
\end{table}

\subsubsection{Podřazená podkapitola}

Vlastní text práce.

\begin{figure}[htb!]
\includegraphics{img/MES.png}
\caption{Název obrázku/grafu/fotografie.}
Zdroj: citace zdroje, nebo autor, vlastní zpracování
%\label{fig:MES}
\end{figure} 


% ################################
\section{Shrnutí a diskuse výsledků}

Souhrn a diskuse vlastních výsledků získaných v průběhu řešení problému. 


% ################################
\section{Závěry a doporučení}

Kritická diskuse nad výsledky, ke kterým autor dospěl (soulad výsledků s literaturou či předpoklady; výsledky a okolnosti, které zvláště ovlivnily předkládanou práci atd.). Je vhodné naznačit i případné další (popř. alternativní) možnosti zkoumání dané problematiky a otevřené problémy pro další studium.


  \iftwoside
    \cleardoublepage
  \fi

  % Použitá literatura / Bibliography
  % ----------------------------------------------
  \begin{thebibliography}{1}
  \bibitem{autor01}
  PŘÍJMENÍ, Jméno. Název knihy : podnázev. Vydání. Místo vydání : Název nakladatelství, rok vydání. Počet stran. ISBN.
  \bibitem{autor02}
  Název. Název odpovědné korporace, instituce. Roky vydání (od-do), ročník (od-do). Místo vydání : Název nakladatelství. Standardní číslo (ISSN).
  \bibitem{autor03}
  Jméno autora. Název zdrojového dokumentu. Označení vydání. Číslo části. Místo vydání : Název nakladatelství, rok vydání. Rozsah díla. Standardní číslo. Lokace ve zdrojovém dokumentu.
  \bibitem{autor04}
  Autor příspěvku. Název příspěvku. In Název zdrojového dokumentu. Primární odpovědnost (autor) za zdrojový dokument. Vydání. Místo vydání : Název nakladatelství, rok. Lokace ve zdrojovém dokumentu.
  \bibitem{autor05}
  Autor. Název článku. Název seriálu, rok vydání, ročník, číslo, strany od-do.
  \bibitem{autor06}
  PŘÍJMENÍ, Jméno autora. Název monografie nebo www stránky (tag "title") : podnázev [druh média]. Vydání. Místo vydání : Vydavatel, datum publikování, datum poslední revize [citováno dne]. <dostupnost - URL adresa>. Standardní číslo.
  \bibitem{autor07}
  PŘÍJMENÍ, Jméno autora. Název zdrojového dokumentu [druh média]. Vydání. Místo vydání : Vydavatel, datum publikování, datum poslední revize [citováno dne]. Označení části nebo kapitoly. Název části nebo kapitoly. <dostupnost -URL adresa>. Standardní číslo.
  \bibitem{autor08}
  PŘÍJMENÍ, Jméno autora příspěvku. Název příspěvku [druh média]. In Název zdrojového dokumentu. Vydání. Místo vydání : Vydavatel, datum publikování, datum poslední revize [citováno dne]. <dostupnost -URL adresa>. Standardní číslo.
  \bibitem{autor09}
  Název konference nebo fóra [druh média]. Místo vydání : Vydavatel, datum vydání [citováno dne]. <dostupnost -URL adresa>.
  \end{thebibliography}

  % nebo BibTeX / or BibTeX

  %\bibliographystyle{iso690}
  %\bibliography{literatura}
  
  \newpage

  % vynechání stránky v oboustranném režimu
  % Skip the page in the two-sided mode
  \iftwoside
    \cleardoublepage
  \fi

  % Přílohy / Appendices
  % ---------------------------------------------

  \ifczech
    \renewcommand{\appendixpagename}{Přílohy}
    \renewcommand{\appendixtocname}{Přílohy}
    \renewcommand{\appendixname}{Příloha}
  \fi
  \ifslovak
    \renewcommand{\appendixpagename}{Prílohy}
    \renewcommand{\appendixtocname}{Prílohy}
    \renewcommand{\appendixname}{Príloha}
  \fi
 
  % Přílohy / Appendices
  \section{\appendixpagename}
  
\appendix

1)
%2)



\newpage

% čísla příloh v záhlaví / appendix numbering in headers
% ------------------------------------------------------
\newcommand{\appendixsection}[1]{\section*{#1} \markboth{}{#1}}
\pagestyle{fancy}
\fancyhead{}
\fancyhead[R]{\rightmark}
\renewcommand{\headrulewidth}{0pt}
\setlength{\headheight}{15pt}
% ------------------------------------------------------

\appendixsection{Příloha č. 1}








%\newpage
%\appendixsection{Příloha č. 2}







% konec příloh / end of appendix
\newpage
\fancyhead{} % zrušení záhlaví / header cancellation

  
  % Zadání / Assignment
  % ---------------------------------------------
  \includepdf{zadani.pdf}
  
\end{document}
